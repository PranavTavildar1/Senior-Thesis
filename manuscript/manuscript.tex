% !TEX TS-program = pdflatexmk
\documentclass[12pt]{article}

%% preamble: Keep it clean; only include those you need
\usepackage{amsmath}
\usepackage[margin = 1in]{geometry}
\usepackage{graphicx}
\usepackage{booktabs}
\usepackage{natbib}

% for space filling
\usepackage{lipsum}
% highlighting hyper links
\usepackage[colorlinks=true, citecolor=blue]{hyperref}


%% meta data

\title{Sentiment Analysis of Twitter in Relation to Fossil Fuel Stock Prices}
\author{Pranav Tavildar\\
  University of Connecticut
}

\begin{document}
\maketitle

%%%%%%%%%%%%%%%%%%%%%%%%%%%%%%%%%%%%%%%%%%%%%%%
\section{Introduction}

\paragraph{Overview of topic}
Fossil fuel companies in the United States spends billions of dollars on advertising in order to sway public sentiment on fossil fuels. This has considerable influence on the way which the public views fossil fuels and in turn might affect stock prices.

\paragraph{Why does it matter}

\paragraph{What has been done}

\paragraph{My Contribution}

\paragraph{Roadmap}

%%%%%%%%%%%%%%%%%%%%%%%%%%%%%%%%%%%%%%%%%%%%%%%

\section{Data Description}

\paragraph{What is the data?}\

\paragraph{how the economic data was obtained}

\paragraph{How the twitter data was scraped}

\paragraph{Exploratory data analysis}

\paragraph{How its going to help answer the research question}

%%%%%%%%%%%%%%%%%%%%%%%%%%%%%%%%%%%%%%%%%%%%%%%

\section{Methods}

\paragraph{Observed Data}

\paragraph{The Transformer Model}

\paragraph{Pearson R-Correlation}

\paragraph{Why we're doing pearson R-Correlation}

\paragraph{assumptions and claims}

%%%%%%%%%%%%%%%%%%%%%%%%%%%%%%%%%%%%%%%%%%%%%%%
\section{Results}
\paragraph{Sentiment Score Analysis}
\paragraph{Correlation}
\paragraph{Visualizations}

%%%%%%%%%%%%%%%%%%%%%%%%%%%%%%%%%%%%%%%%%%%%%%%
\section{Discussion}
\paragraph{Summation of Contributions of the Research}
\paragraph{Limitations of Study}
\paragraph{Future Directions}
This project will require a significant amount of development and expansion before analysis can take place but analyzing the effect of public sentiment on stock prices of fossil fuel companies can provide valuable insight into how public opinion can be used to augment or reduce the effects of climate change.


%\bibliography{propreferences}
%\bibliographystyle{chicago}

\end{document}