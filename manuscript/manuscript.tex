% !TEX TS-program = pdflatexmk
\documentclass[12pt]{article}

%% preamble: Keep it clean; only include those you need
\usepackage{amsmath}
\usepackage[margin = 1in]{geometry}
\usepackage{graphicx}
\usepackage{booktabs}
\usepackage{natbib}

% for space filling
\usepackage{lipsum}
% highlighting hyper links
\usepackage[colorlinks=true, citecolor=blue]{hyperref}


%% meta data

\title{Sentiment Analysis of Twitter in Relation to Fossil Fuel Stock Prices}
\author{Pranav Tavildar\\
  University of Connecticut
}

\begin{document}
\maketitle

%%%%%%%%%%%%%%%%%%%%%%%%%%%%%%%%%%%%%%%%%%%%%%%
\section{Introduction}

%\paragraph{Overview of topic}
	The Chevron Corporation is an America-based energy company thats has operations in over 180 countries where it has vertically integrated all operations in its supply chain such as exploration, production, refinement, chemical manufacturing, transportation, sales, marketing, and power generation. A lot of revenue comes from these streams but they also receive funding from government subsidies and stock revenue. As of when this paper was written, the Chevron Corporation has the 11th highest market capitalization with 361.57 billion dollars in stocks.
	
	As a publicly traded company, Chevron's share price is determined by supply and demand in the market. Generally, this is influenced by factors such as market dynamics, economic conditions and changes to economic policy. This information can likely be derived through public speculation and one of the best places to observe this is Twitter, the "virtual public square."
	
%\paragraph{Review of the Literature}
	There exists a machine learning technique known as sentiment analysis which uses natural language processing to determine whether the sentiment towards any subject is positive or negative and assigns it a score. The preferred method of analyzing sequence data for most natural language processing is through the use of transformer models. This type of model was first introduced in the paper "Attention is all you need" \cite{}
\paragraph{My Contribution}
	In this paper, I seek to use a Bert-Based Transformer model to generate sentiment scores of twitter data from the year 2021 and compare with the 2021 stock prices of the Chevron Corporation (CVX) to determine whether there exists any correlation. 
\paragraph{Roadmap}

%%%%%%%%%%%%%%%%%%%%%%%%%%%%%%%%%%%%%%%%%%%%%%%

\section{Data Description}
In order to begin sentiment analysis in order to solve this problem.
\paragraph{What is the data?}
There are two major methods that were used to obtain the data
\paragraph{how the economic data was obtained}

\paragraph{How the twitter data was scraped}

\paragraph{Exploratory data analysis}
yea 
\paragraph{How its going to help answer the research question}

%%%%%%%%%%%%%%%%%%%%%%%%%%%%%%%%%%%%%%%%%%%%%%%

\section{Methods}

\paragraph{Observed Data}

\paragraph{The Transformer Model}

\paragraph{Pearson R-Correlation}

\paragraph{Why we're doing Pearson R-Correlation}

\paragraph{assumptions and claims}

%%%%%%%%%%%%%%%%%%%%%%%%%%%%%%%%%%%%%%%%%%%%%%%
\section{Results}
\paragraph{Sentiment Score Analysis}
\paragraph{Correlation}
\paragraph{Visualizations}

%%%%%%%%%%%%%%%%%%%%%%%%%%%%%%%%%%%%%%%%%%%%%%%
\section{Discussion}
\paragraph{Summation of Contributions of the Research}
\paragraph{Limitations of Study}
\paragraph{Future Directions}
This project will require a significant amount of development and expansion before analysis can take place but analyzing the effect of public sentiment on stock prices of fossil fuel companies can provide valuable insight into how public opinion can be used to augment or reduce the effects of climate change.


%\bibliography{propreferences}
%\bibliographystyle{chicago}

\end{document}